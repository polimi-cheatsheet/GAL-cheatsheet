\section{Posizione reciproca tra rette-piani}
Definiamo $n = $ numero variabili, $r = \rk A$ e $s = \rk A|\vec{b}$

Nel piano:
\begin{tabular}{@{}lllll@{}}
	$n = 2$ & $r = 2$ & $s = 3$ & Imp & Più punti di intersezione \\
	        & $r = 2$ & $s = 2$ & Det & Un punto di intersezione \\
	        & $r = 1$ & $s = 2$ & Imp & Rette parallele \\
	        & $r = 1$ & $s = 1$ & Ind & Rette coincidenti \\
\end{tabular}
Nello spazio:
\begin{tabular}{@{}lllll@{}}
	$n = 2$ & $r = 2$ & $s = 3$ & Imp & Retta + piani $\parallel$ ad essa \\
			& $r = 2$ & $s = 2$ & Det & Una retta \\
			& $r = 1$ & $s = 2$ & Imp & Piani paralleli \\
			& $r = 1$ & $s = 1$ & Ind & Piani coincidenti \\
	$n = 3$ & $r = 3$ & $s = 4$ & Imp & Più punti di intersezione \\
			& $r = 3$ & $s = 3$ & Det & Stella di piani \\
			& $r = 2$ & $s = 3$ & Imp & Retta + piani $\parallel$ ad essa \\
			& $r = 2$ & $s = 2$ & Ind & Una retta \\
			& $r = 1$ & $s = 2$ & Ind & Piani paralleli \\
			& $r = 1$ & $s = 1$ & Ind & Piani coincidenti \\
\end{tabular}
