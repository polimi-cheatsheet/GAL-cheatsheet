\subsection{Verifica similitudine}
\begin{tabular}{cc}
	$
	A_h = \left[
		\arraycolsep=2.0pt\def\arraystretch{1.0}
		\begin{array}{@{}ccc@{}}
			1   & 2 & 0 \\
			h+1 & 1 & 0 \\
			0   & 0 & 2 \\
		\end{array}
	\right]
	$ &
	$
	B = \left[
		\arraycolsep=2.0pt\def\arraystretch{1.0}
		\begin{array}{@{}ccc@{}}
			1 & 0 & 0 \\
			0 & 1 & 0 \\
			0 & 0 & 2 \\
		\end{array}
	\right]
	$
\end{tabular}

$\det A = -4h-2 = \det B = 2 \Rightarrow h = -1$

$p_A(\lambda) = (2-\lambda)(1+\sqrt{2(h+1)}-\lambda)(1-\sqrt{2(h+1)}-\lambda)$
$p_B(\lambda) = (2-\lambda)(1-\lambda)(1-\lambda)$

$A$ e $B$ devono avere gli stessi autovalori $\Rightarrow h = -1$

%Studio diagonalizzabilità di $A_{-1} =
%\left[
%	\arraycolsep=2.0pt\def\arraystretch{1.0}
%	\begin{array}{@{}ccc@{}}
%		1 & 2 & 0 \\
%		0 & 1 & 0 \\
%		0 & 0 & 2 \\
%	\end{array}
%\right]
%$
%
%$\lambda_1 = 2$ è semplice e quindi regolare. Studio $\lambda_2 = 1$ \\
%$V_1 = \ker 
%\left[
%	\arraycolsep=2.0pt\def\arraystretch{1.0}
%	\begin{array}{@{}ccc@{}}
%		0 & 2 & 0 \\
%		0 & 0 & 0 \\
%		0 & 0 & 1 \\
%	\end{array}
%\right]
%$
%$\Rightarrow \dim V_1 = 1 \neq m_a(1)$
%$\Rightarrow \lambda_2$ non è regolare.

$A_{-1}$ non è diagonalizzabile, non è simile a $B$ (diagonale).
