\section{Matrici ortogonali}

U è ortogonale se è reale e $U^TU=I_n$:

\begin{tabular}{lll}
	$\det U = \pm1$ & $U^{-1}=U^T$ & $\lambda = \pm 1$
\end{tabular}

$f$ rappresentata da $U$ orto. è un'isometria (conserva la norma).

\begin{tabular}{lll}
	In $\mathbb{R}^2$ &
	$
		M_1 = \begin{bmatrix}
			\cos\theta & -\sin\theta \\
			\sin\theta & \cos\theta \\
		\end{bmatrix}
	$ &
	$
		M_2 = \begin{bmatrix}
			\cos\theta & \sin\theta \\
			\sin\theta & -\cos\theta \\
		\end{bmatrix}
	$
\end{tabular}

$M_1$ ($\det = 1$) rappresenta una rotazione del piano di $\theta$. $M_2$ ($\det = -1$) rappresenta una simmetria rispetto $V_1 = \Lin(\cos\frac{\theta}{2}, \sin\frac{\theta}{2})$.

Se $U$ è orto. simm. allora rappr. una simm. orto. rispetto $V_1$. $V_1$ rimane fermo, $V_{-1}$ viene invertito. $U=I-2P_{-1}$