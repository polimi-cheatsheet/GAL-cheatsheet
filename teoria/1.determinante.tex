\section{Determinante}
$\det: M_{n,n}(K) \rightarrow K$

\begin{tabularx}{\textwidth}{lX}
	$n = 1$ & $A = [a] \quad \det{A} = a$ \\
	$n > 1$ &
	Ricorsivamente \newline
	$A_{ik}$ ottenuta da $A$ togliendo riga $i$ e colonna $k$ \newline
	$M_{ik} = \det A_{ik}$ (detto minore complementare) \newline
	$C_{ik} = (-1)^{i+k}M_{ik}$ (detto complemento algebrico) \newline
	$\det A = \sum_{i=1}^{n} a_{1i}C_{1i}$ \\
\end{tabularx}

I th. di Laplace: si può usare una riga o una colonna qualsiasi.

\begin{tabular}{l}
	$\det A = \det A^T$ \\
	Se una riga è di zeri: $\det A = 0$ \\
	Se si scambiano 2 righe: $\det A' = -\det A$ \\
	Se due righe parallele sono proporzionali: $\det A = 0$ \\
	Moltiplicando una riga per $t$: $\det A' = t\det A$ \\
	$\det tA = t^n\det A$ \\
	In una matrice triangolare: $\det A = \prod_{i=1}^{n} a_{ii}$ \\
	T. di Binet: $\det AB = \det A \cdot \det B$
\end{tabular}

Regola di Kronecker: se esiste una sottomatrice quadrata $A'_{n,n}$ con $\det A' \ne 0$ allora $\rk A \ge n$.
Se tutte le matrici ottenute per orlatura da $A'$ hanno $\det = 0$ allora $\rk A = n$.

\begin{tabular}{cc}
	$A = \begin{bmatrix}
	a & b \\
	c & d \\
	\end{bmatrix}$ &
	$B = \begin{bmatrix}
	a & b & c \\
	d & e & f \\
	g & h & i \\
	\end{bmatrix}$ \\
	$\det A = ad - bc$ &
	$\det B = aei + bfg + cdh - ceg - afh - bdi$ \\
\end{tabular}