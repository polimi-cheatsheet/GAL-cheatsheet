\subsection{Diagonalizzazione di una matrice}
$
A = \left[
	\arraycolsep=2.0pt\def\arraystretch{1.0}
	\begin{array}{@{}ccc@{}}
		1 & 1 & 0 \\
		0 & k & 0\\
		0 & 0 & 2\\
	\end{array}
\right]
$
%\hspace{1cm}
\begin{tabular}{l}
	$p(\lambda) = \det(A-\lambda I) = (1-\lambda)(k-\lambda)(2-\lambda)$ \\
	$\lambda_1 = 1, \lambda_2 = 2, \lambda_3 = k$
\end{tabular}

Se $k\neq1$ e $k\neq2 \Rightarrow \lambda_1, \lambda_2, \lambda_3$ sono semplici e quindi regolari \\
$\quad\Rightarrow$ A è diagonalizzabile

% This is the "beauty" of LaTeX
Se $k = 1 \Rightarrow \lambda_2$ è semplice, studio $\lambda_1$: \\
$\quad V_1 = \ker(A-\lambda_1I) = \ker
\left(
	\left[
		\arraycolsep=2.0pt\def\arraystretch{1.0}
		\begin{array}{@{}ccc@{}}
			1 & 1 & 0 \\
			0 & 1 & 0\\
			0 & 0 & 2\\
		\end{array}
	\right] - I 
\right) = \ker
\left[
	\arraycolsep=2.0pt\def\arraystretch{1.0}
	\begin{array}{@{}ccc@{}}
		0 & 1 & 0 \\
		0 & 0 & 0\\
		0 & 0 & 1\\
	\end{array}
\right]$ \\
$\quad \Rightarrow \dim(V_1) = 1 \Rightarrow m_g \neq m_a$
$\ \Rightarrow \lambda_1$ non è regolare \\
$\quad \Rightarrow$ A non è diagonalizzabile


Se $k = 2 \Rightarrow \lambda_1$ è semplice, studio $\lambda_2$:\\
$\quad V_2 = \ker(A-\lambda_2I) = \ker
\left(
	\left[
		\arraycolsep=2.0pt\def\arraystretch{1.0}
		\begin{array}{@{}ccc@{}}
			1 & 1 & 0 \\
			0 & 2 & 0\\
			0 & 0 & 2\\
		\end{array}
	\right] - 2I 
\right) = \ker
\left[
	\arraycolsep=2.0pt\def\arraystretch{1.0}
	\begin{array}{@{}ccc@{}}
		-1 & 1 & 0 \\
		0 & 0 & 0\\
		0 & 0 & 0\\
	\end{array}
\right]$ \\
$\quad \Rightarrow \dim(V_2) = 2 \Rightarrow m_g = m_a$
$\ \Rightarrow \lambda_2$ è regolare \\
$\quad \Rightarrow$ A è diagonalizzabile
