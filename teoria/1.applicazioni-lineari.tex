\section{Applicazioni lineari}
$f: V \rightarrow W \quad f(v_1+v_2) = f(v_1) + f(v_2) \quad f(kv) = kf(v)$

\begin{tabular}{l@{\quad}l}
	Definizione & $f(k_1v_1 + k_2v_2) = k_1f(v_1) + k_2f(v_2)$ \\
	Nucleo & $\ker f = \{ \vec{v} \in V | f(\vec{v}) = \vec{0}_{W} \}$ \\
		   & $\ker f$ è un sottospazio di V e $\vec{0} \in \ker f$ \\
	Fibra di $\vec{w}$ & $\{ \vec{v} \in V | f(\vec{v}) = \vec{w} \}$ \\
	Immagine & $\Imm f = f(V) = \Lin(C_1, C_2, ..., C_n)$ \\
			 & $f(\vec{0}_V) = \vec{0}_W$ \\
	Forma lineare & $f: V \rightarrow K$ \\
	Endomorfismo & $f: V \rightarrow V$ \\
	Iniettiva & $\ker f = \{\vec{0}\} \quad \rk A = n$ \\
	Suriettiva & $\Imm f = W \quad \rk A = m$ \\
	Biunivoca & Iniettiva + suriettiva (isomorfismo) $\det A \ne 0$ \\
	Somma di f.l. & $(f+g)(\vec{v}) = f(\vec{v}) + g(\vec{v})$ \\
	Scalare per f.l. & $(hf)(\vec{v}) = hf(\vec{v}) \qquad h \in K$ \\
\end{tabular}

T. nullità più rango: $|V| = |\ker f| + |\Imm f|$
